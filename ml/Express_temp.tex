\documentclass[10pt]{article}
%% Specify the Express journal you are submitting to
%\usepackage[OME]{express}
\usepackage[OE]{express}
%\usepackage[BOE]{express}

\begin{document}
\title{Statement of Purpose}

\author{Fabio Votta}

\address{University of Stuttgart, M.A.: Empirical Social and Political Research, \href{fabio.votta@gmail.com}{fabio.votta@gmail.com}}

%\email{\authormark{*}opex@osa.org} %% email address is required

% \homepage{http:...} %% author's URL, if desired

%%%%%%%%%%%%%%%%%%% abstract and OCIS codes %%%%%%%%%%%%%%%%
%% [use \begin{abstract*}...\end{abstract*} if exempt from copyright]

%%%%%%%%%%%%%%%%%%%%%%%%%%  body  %%%%%%%%%%%%%%%%%%%%%%%%%%

\vspace{0.4cm}
\begin{center}
\noindent \textbf{\large Application for Ph.D. Program in Politics at the \textsc{University of Princeton}}
\end{center}
\vspace{0.2cm}

\noindent Dear Members of the selection committee,

\vspace{0.3cm}

at the end of this summer I will obtain a master’s degree in political science with a major in empirical methods. After this diploma, my prior objective is to pursue an academic career within the field of computational social sciences. In this perspective, your graduate school clearly does not only constitute an opportunity to deepen my methodological skills but is also one of the few German universities that produces quality research in computational social science. As programming was not the focus in Stuttgart, I had to wait for my internship at GESIS Mannheim in 2014 to discover the programming language R. Convinced of the importance and research-potential of such a tool for social scientists, as soon as I cam back to Stuttgart, I founded the R-User-GroupStuttgart and committed myself to train students to the use of open source software. Within this informal context, I taught beginner courses on data preparation and visualization as well as advanced workshops on predictive modeling and Shiny App development. I enjoy teaching statistics and above all R.

\vspace{0.3cm}

 My latest source of inspiration was the UseR!2017 in Brussels. Beyond these technical skills, I was always fully aware, that tools are not much without a theoretical setting. Thus, I always kept in mind to stay in touch with theoretical debates, despite my clear technical focus. To do so, I worked with Prof. André Bächtiger on several projects investigating democratic preferences within experimental settings. On theoretical point of view, I am fascinated by the deliberative theory, whose empirical research is a topic I would like to deepen during my PhD. These collaborations with Prof. Bächtiger and other students allowed me to gather some academic experience. For instance, my participation as a presenter to the ECPR General Conference in Prague (2016) thought me lot about how academia works. In Mai, I expect to gather even more experience through a presentation at VOXPOL conference on Combating Online Extremism in Oxford coming this May. The goal of this paper is to identify Alt-Right talking points from Twitter, Facebook and YouTube content. First, a Shiny App was developed that provides a user interface for reliably coding different indicators like racism, anti-sematism and misogyny. 15 coders voluntarily contributed sofar. Second, we use Recurrent Neural Networks to predict those classes by learning sequences of words and their association to the output. This type of model architecture yields superior accuracy for all types of NLP problems. 
 
 \vspace{0.3cm}
 
 As mentioned earlier, I would really like to dedicate my PhD to the empirical research on deliberation. Most particularly, I believe that recent development in methods on Natural Language Processing might be very helpful. Indeed deliberation relies essentially on the exchange of unstructured information – language - which have been so far almost impossible to study. However, the progress in computational power changed the rule of the game and allow the automatic analysis of huge and extremely complex source of data. With such a research-agenda, I believe I could truly benefit from the interdisciplinary dynamic offered by the GSDS and I would be very enthusiastic to join your Graduate School as a PhD student. Hopefully you will be convinced that my university training, coupled with my volunteer experience, have allowed me to acquire the required qualities to conduct PhD research. Thank you for your consideration. Yours truly, Simon Roth

\end{document}
